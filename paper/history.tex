\section{A Brief History of Arbitrary Code Execution}

Arbitrary code execution has been around since assembly programming was commonplace. At its simplest form, arbitrary code execution is a software bug, not the result of a brute-force attack on a resource, such as trying millions of passwords on a particular website to steal the information of a particular user. That is, arbitrary code execution is, theoretically, preventable.

Because arbitrary code execution has been a software engineering issue for so long, it comes in many forms. Some of the most famous examples of arbitrary code execution were a result of bugs in standard C programming libraries, with \texttt{strcpy} being the most famous. Moreover, the \texttt{fingerd} daemon in Linux was known to have a vulnerability that allowed buffer overflow, and consequently, any user to execute local binaries.

Today, the web is perhaps the most common place for arbitrary code execution to occur. In 2012, the Foxypress Wordpress plugin was found to have an exploit that could allow arbitrary code execution. One of the plugin’s source code files, \texttt{uploadify}, was responsible for allowing users to upload remote files to their Wordpress servers. When the exploit was found, it allowed for remote files to execute malicious PHP code on the user’s Wordpress server. Potential outcomes of this bug include website vandalism, database corruption, and stealing of credit card (or any other) information. It was found that versions 0.4.1.1 - 0.4.2.1 were vulnerable (1). The source code containing the exploit can be seen in Figure 1.

Furthermore, In June 2016, it was found that Google Chrome’s PDF viewer had the potential for arbitrary code execution. In CVE-2016-1681, found by Aleksandar Nikolic, it was discovered that if a PDF has an embedded \texttt{jpeg2000} image could trigger a heap buffer overflow. The simple fix, according to Nikolic, was to change an \texttt{assert} statement to an \texttt{if} statement in the call to a  OpenJPEG library function that Google used in their PDF viewer. (4)

Lastly, Apple discovered a vulnerability in \textt{configd}, a DNS networking daemon that is included on devices running iOS, their mobile operating system. A heap based buffer overflow issue existed in the DNS client library. A malicious application with the ability to spoof responses from the local \texttt{configd} service may have been able to cause arbitrary code execution in DNS clients. This bug was reported by PanguTeam, a team of hackers responsible for many versions of the programs that allowed users to “jailbreak” their iOS devices. (5)