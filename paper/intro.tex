\section{Introduction}

Arbitrary code execution is a type of injection exploit in software, relating to many well-known exploit concepts, including Shellshock, email injection, and cross-site scripting. Arbitrary code execution refers to the process by which one or more processes running in a program are compromised, allowing code to be ran that wasn't intended to be. Traditionally, it's assumed the goal of arbitrary code execution is a malicious one, such as the theft of information or corruption of data, but that's not always the case. There are instances where arbitrary code execution are used for research and entertainment purposes, such as finding ways to beat video games faster than would be possible under normal conditions, or to generate an entirely new experience using the game's assets. In any case, arbitrary code execution is an evergreen topic because of the impact is has on the software industry, with security in software being paramount. In this paper, we will discuss several aspects of arbitrary code execution, citing and explaining examples along the way.